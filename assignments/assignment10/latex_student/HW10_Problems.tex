
\newcommand{\VAESolAa}{
%%%%%%%%%%%%%%%%%%%%%%%%%%%%%%%%%%%%
%%
%%.   YOUR SOLUTION FOR PROBLEM A BELOW THIS COMMENT
%%
%%%%%%%%%%%%%%%%%%%%%%%%%%%%%%%%%%%%
$$D_{KL}(p(x) \| q(x)) = - \int p(x) \bigg( \log \bigg\{ \frac{q(x)}{p(x)} \bigg\} \bigg)dx$$
}

\newcommand{\VAESolAb}{
%%%%%%%%%%%%%%%%%%%%%%%%%%%%%%%%%%%%
%%
%%.   YOUR SOLUTION FOR PROBLEM A BELOW THIS COMMENT
%%
%%%%%%%%%%%%%%%%%%%%%%%%%%%%%%%%%%%%
$$KL(p(x) \| q(x)) = - \int p(x) \bigg( \log \bigg\{ \frac{q(x)}{p(x)} \bigg\} \bigg)dx \geq - \log \int p(x) \frac{q(x)}{p(x)} = - \log \int q(x)dx = 0$$
}


\newcommand{\VAESolBa}{	
%%%%%%%%%%%%%%%%%%%%%%%%%%%%%%%%%%%%
%%
%%.   YOUR SOLUTION FOR PROBLEM B BELOW THIS COMMENT
%%
%%%%%%%%%%%%%%%%%%%%%%%%%%%%%%%%%%%%
\begin{align*}
\log p_\theta(x) &= \int_z q_\phi(z|x)\log\frac{p_\theta(x,z)}{q_\phi(z|x)}dz + \int_z q_\phi(z|x)\log\frac{q_\phi(z|x)}{p_\theta(z|x)}dz\\
&= \mathcal{L}(p_{\theta}, q_{\phi}) + D_{KL}(q_{\phi}, p_{\theta})\\
&\geq \mathcal{L}(p_{\theta}, q_{\phi})
\end{align*}

}

\newcommand{\VAESolBb}{	
%%%%%%%%%%%%%%%%%%%%%%%%%%%%%%%%%%%%
%%
%%.   YOUR SOLUTION FOR PROBLEM B BELOW THIS COMMENT
%%
%%%%%%%%%%%%%%%%%%%%%%%%%%%%%%%%%%%%
$\mathcal{L}(p_{\theta}, q_{\phi})$ is often referred to as empirical lower bound (ELBO).
}

\newcommand{\VAESolBc}{	
%%%%%%%%%%%%%%%%%%%%%%%%%%%%%%%%%%%%
%%
%%.   YOUR SOLUTION FOR PROBLEM B BELOW THIS COMMENT
%%
%%%%%%%%%%%%%%%%%%%%%%%%%%%%%%%%%%%%
It holds with equality if and only if $q_{\phi} = p_{\theta}$ for all x.
}


\newcommand{\VAESolC}{
%%%%%%%%%%%%%%%%%%%%%%%%%%%%%%%%%%%%
%%
%%.   YOUR SOLUTION FOR PROBLEM C BELOW THIS COMMENT
%%
%%%%%%%%%%%%%%%%%%%%%%%%%%%%%%%%%%%%
\begin{align*}
	D_{KL}(q(z|x) \|p(z)) &= \mathbb{E}[\log q(z|x) - p(z)]\\
	&= \frac{1}{2} \cdot (\sigma_z^2 + \mu_z^2 - 1 - \log (\sigma_z^2))
\end{align*}
}


\newcommand{\VAESolD}{
%%%%%%%%%%%%%%%%%%%%%%%%%%%%%%%%%%%%
%%
%%.   YOUR SOLUTION FOR PROBLEM D BELOW THIS COMMENT
%%
%%%%%%%%%%%%%%%%%%%%%%%%%%%%%%%%%%%%
\\It is more numerically stable to take exponent compared to computing log.
$$D_{KL}(q(z|x) \|p(z)) = \frac{1}{2} \cdot (\exp (\sigma_z^2) + \mu_z^2 - 1 - \sigma_z^2)$$

Furthermore, if we model just $\sigma_z^2$, it always outputs a non-negative number, but if we model $\sigma_z^2$ in log space, we can get all the number in $\mathbb{R}^1$, which will give us more options.
}

\newcommand{\VAESolE}{
%%%%%%%%%%%%%%%%%%%%%%%%%%%%%%%%%%%%
%%
%%.   YOUR SOLUTION FOR PROBLEM E BELOW THIS COMMENT
%%
%%%%%%%%%%%%%%%%%%%%%%%%%%%%%%%%%%%%
In a nutshell, reparametrization trick make sure we can backpropagate. We are given $z$ that is drawn from a distribution $q_\phi (z \vert x)$, and we want to take derivatives of a function of $z$ with respect to $\phi$, The reparametrization trick lets us backpropagate (take derivatives using the chain rule) with respect to $\phi$ through the objective (the ELBO) which is a function of samples of the latent variables $z$.
}


 %This file will not be provided to students since it contains the solution

% Problem Explanation:
% - first argument is the number of points
% - second argument is the title and the text
 \examproblem{2}{KL Divergence
}

%%%%%%%%%%%%%%%%%%%%%%%%%%%%%%%%%%%%%%
%%%%%  BEGINNING OF SUBPROBLEMS LIST
 \begin{enumerate}

% Subproblem description
 \examproblempart{[1 point] What is the expression of the KL divergence $D_{KL}(q(x)||p(x))$ given two continuous distributions $p(x)$ and $q(x)$ defined on the domain of $\mathbb{R}^1$? \\}
 
\bookletskip{0.0}   %in inches

% Solution box 
 \framebox[14.7cm][l]{
 \begin{minipage}[b]{14.7cm}
 \answer{ \VAESolAa}
 \end{minipage}
 }



% Subproblem description
\examproblempart{[1 point] Show that the KL divergence is non-negative. You can use Jensen's inequality here without proving it.\\}

\bookletskip{0.0}   %in inches

% Solution box 
\framebox[14.7cm][l]{
	\begin{minipage}[b]{14.7cm}
		\answer{ \VAESolAb}
	\end{minipage}
}

\end{enumerate}


 \examproblem{3}{In the class, we derive the following equality:
 	$$
 	\log p_\theta(x) = \int_z q_\phi(z|x)\log\frac{p_\theta(x,z)}{q_\phi(z|x)}dz + \int_z q_\phi(z|x)\log\frac{q_\phi(z|x)}{p_\theta(z|x)}dz 
 	$$ 
 	Instead of maximizing the log likelihood $\log p_\theta(x)$ w.r.t.~$\theta$, we find a lower bound for $\log p_\theta(x)$ and maximize the lower bound. \\
}

%%%%%%%%%%%%%%%%%%%%%%%%%%%%%%%%%%%%%%
%%%%%  BEGINNING OF SUBPROBLEMS LIST
\begin{enumerate}


% Subproblem description 
 \examproblempart{[1 point] Use the above equation and your result in 1(b) to give a lower bound for $\log p_\theta(x)$. 
}

\bookletskip{0.0}   %in inches

% Solution box 
 \framebox[14.7cm][l]{
 \begin{minipage}[b]{14.7cm}
 \answer{\VAESolBa}
 \end{minipage}
 }


% Subproblem description 
\examproblempart{[1 point] What do people usually call the bound?\\
}

\bookletskip{0.0}   %in inches

% Solution box 
\framebox[14.7cm][l]{
	\begin{minipage}[b]{14.7cm}
		\answer{\VAESolBb}
	\end{minipage}
}

% Subproblem description 
\examproblempart{[1 point] In what condition will the bound be tight?\\
}

\bookletskip{0.0}   %in inches

% Solution box 
\framebox[14.7cm][l]{
	\begin{minipage}[b]{14.7cm}
		\answer{\VAESolBc}
	\end{minipage}
}


 \end{enumerate}

 \examproblem{2}{Given $z\in \mathbb{R}^1$, $p(z) \sim \mathcal{N}(0,1)$ and $q(z|x)\sim\mathcal{N}(\mu_z,\sigma^2_z)$, write $D_{KL}(q(z|x)||p(z))$ in terms of $\sigma_z$ and $\mu_z$. \\
}


\bookletskip{0.0}   %in inches

% Solution box 
\framebox[14.7cm][l]{
	\begin{minipage}[b]{14.7cm}
		\answer{\VAESolC}
	\end{minipage}
}



 \examproblem{1}{In VAEs, the encoder computes the mean $\mu_z$ and the variance $\sigma_z^2$ of $q_\phi(z|x)$ assuming $q_\phi(z|x)$ is Gaussian. Explain why we usually model $\sigma_z^2$ in log space, i.e., modeling $\log\sigma_z^2$ instead of $\sigma_z^2$ when implementing it using neural nets?\\
}

\bookletskip{0.0}   %in inches

% Solution box 
\framebox[14.7cm][l]{
	\begin{minipage}[b]{14.7cm}
		\answer{\VAESolD}
	\end{minipage}
}


 
  \examproblem{1}{Why do we need the reparameterization trick when training VAEs instead of directly sampling from the latent distribution $\mathcal{N}(\mu_z,\sigma_z^2)$? \\
 }
 

\bookletskip{0.0}   %in inches

% Solution box 
\framebox[14.7cm][l]{
	\begin{minipage}[b]{14.7cm}
		\answer{\VAESolE}
	\end{minipage}
}


\bookletskip{0.0}   %in inches

